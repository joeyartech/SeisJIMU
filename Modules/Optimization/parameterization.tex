\documentclass[9pt]{beamer}
\include{config}

\usepackage{amssymb}
\usepackage{textcomp}

\definecolor{usccardinal}{rgb}{0.6, 0.0, 0.0}
\definecolor{bostonuniversityred}{rgb}{0.8, 0.0, 0.0}

\newcommand{\dxdt}{\mathrm{d}x\mathrm{d}t}
\newcommand{\diff}{\mathrm{d}}
\newcommand{\iunit}{\mathrm{i}}
\newcommand{\e}[1]{\mathrm{e}^{#1}}
\newcommand{\expi}[1]{\mathrm{e}^{\mathrm{i}#1}}
\newcommand{\C}{\mathbb{C}}
\renewcommand{\L}{\mathcal{L}}
\newcommand{\I}{\mathcal{I}}
\newcommand{\partderi}[2]{\frac{\partial#1}{\partial#2}}

\newcommand{\red}{\color{red}}
\newcommand{\bur}{\color{bostonuniversityred}}
\newcommand{\blue}{\color{blue}}
\newcommand{\obs}{^\text{obs}}
\renewcommand{\cal}{^\text{cal}}
\newcommand{\comp}{^\text{comp}}
\newcommand{\true}{^\text{true}}
\newcommand{\bas}{_\text{bas}}
\newcommand{\mon}{_\text{mon}}

% \graphicspath{{Fig/}}

%--------------------------------------------------------------------------------------
% \titlegraphic{\includegraphics[scale=0.5]{Fig/logo.pdf}\;\;\;\includegraphics[scale=0.2]{Fig/logo-ISTerre.png}}
\title[]{Fr\'echet derivatives under various parameterizations \\ using chain rule}
\subtitle{for LEGO module m\_parameterization.f90}
\author[joey]{joeyartech\\
\scriptsize{joeywzhou1986@gmail.com}}
\institute[]{Department of Geosciences, The University of Texas at Dallas}
%\date[\today]{}
%--------------------------------------------------------------------------------------


\begin{document}
%\small

\maketitle \clearpage

% \frame{
% \frametitle{Outline}
% \small{
% \tableofcontents[section,subsectionstyle=hide/show/show] }}

\begin{frame}\frametitle{Overview}
  \framebox{WaveEq lm $\Rightarrow$ optim x,g}
  
  \begin{itemize}
    \item WaveEq uses lm and produces gkpa, glda, gmu and grho under moduli-density parameterization.
    \item m\_parameterization.f90 converts them to user-specified parameterization, which can be
    \begin{itemize}
      \item moduli-density (kpa, lda, mu, rho)
      \item velocities-density (vp, vs, rho)
      \item velocities-impedance (vp, vs, Ip)
      \item slowness-density (sp, sps, rho)
    \end{itemize}
    \item Also we may consider ``passive parameters'' for hard constraints (e.g. Gardner) between these parameters.
  \end{itemize}

\end{frame}

\section{velocities-density}

\begin{frame}\frametitle{PARAMETERIZATION velocities-density}
\framesubtitle{isotropic ACoustic}
  
  \begin{minipage}{0.4\linewidth}
    \[\left\{ \begin{array}{l}
      \kappa = \rho V_P^2 \\
      \rho_0 = \rho 
    \end{array} \right.\]
    
    \begin{eqnarray}
      \nabla_{V_P} &=& \nabla_{\kappa} \partderi{\kappa}{V_P} + \nabla_{\rho_0} \partderi{\rho_0}{V_P} \nonumber\\
                   &=& \nabla_{\kappa} 2\rho V_P \nonumber\\
      \nabla_{\rho} &=& \nabla_{\kappa} \partderi{\kappa}{\rho} + \nabla_{\rho_0} \partderi{\rho_0}{\rho} \nonumber\\
		    &=& \nabla_{\kappa} V_P^2 + \nabla_{\rho_0} \nonumber
    \end{eqnarray}

  \end{minipage} \vline
  \begin{minipage}{0.4\linewidth}
    \[\left\{ \begin{array}{l}
      V_P  = \sqrt{\kappa/\rho_0} \\
      \rho = \rho_0
    \end{array} \right.\]
    
    \begin{eqnarray}
      \nabla_{\kappa} &=& \nabla_{V_P} \partderi{V_P}{\kappa} + \nabla_{\rho} \partderi{\rho}{\kappa} \nonumber\\
                      &=& \nabla_{V_P} 0.5(\kappa/\rho)^{0.5}/\rho \nonumber\\
      \nabla_{\rho_0} &=& \nabla_{V_P} \partderi{V_P}{\rho_0} + \nabla_{\rho} \partderi{\rho}{\rho_0} \nonumber\\
                      &=& \nabla_{V_P} (-0.5)(\kappa/\rho)^{0.5} \rho^{-2} + \nabla_{\rho} \nonumber
    \end{eqnarray}
  \end{minipage}

\end{frame}

% \begin{frame}\frametitle{PARAMETERIZATION velocities-density + PASSIVE gardner}
% \framesubtitle{isotropic ACoustic}
%   
%   \begin{minipage}{0.4\linewidth}
%     \[\left\{ \begin{array}{l}
%       \kappa = \rho V_P^2 \\
%       \rho_0 = \rho 
%     \end{array} \right.\]
%     
%     \begin{eqnarray}
%       \nabla_{V_P} &=& \nabla_{\kappa} \partderi{\kappa}{V_P} + \nabla_{\rho_0} \partderi{\rho_0}{V_P} \nonumber\\
%                    &=& \nabla_{\kappa} 2\rho V_P \nonumber\\
%       \nabla_{\rho} &=& \nabla_{\kappa} \partderi{\kappa}{\rho} + \nabla_{\rho_0} \partderi{\rho_0}{\rho} \nonumber\\
% 		    &=& \nabla_{\kappa} V_P^2 + \nabla_{\rho_0} \nonumber
%     \end{eqnarray}
% 
%   \end{minipage} \vline
%   \begin{minipage}{0.4\linewidth}
%     \[\left\{ \begin{array}{l}
%       V_P  = \sqrt{\kappa/\rho_0} \\
%       \rho = \rho_0
%     \end{array} \right.\]
%     
%     \begin{eqnarray}
%       \nabla_{\kappa} &=& \nabla_{V_P} \partderi{V_P}{\kappa} + \nabla_{\rho} \partderi{\rho}{\kappa} \nonumber\\
%                       &=& \nabla_{V_P} 0.5(\kappa/\rho)^{0.5}/\rho \nonumber\\
%       \nabla_{\rho_0} &=& \nabla_{V_P} \partderi{V_P}{\rho_0} + \nabla_{\rho} \partderi{\rho}{\rho_0} \nonumber\\
%                       &=& \nabla_{V_P} (-0.5)(\kappa/\rho)^{0.5} \rho^{-2} + \nabla_{\rho} \nonumber
%     \end{eqnarray}
%   \end{minipage}
% 
% \end{frame}
% 
% \begin{frame}\frametitle{PARAMETERIZATION velocities-density}
% \framesubtitle{P-SV}
%   
%   \begin{minipage}{0.4\linewidth}
%     \[\left\{ \begin{array}{l}
%       \lambda = \rho (V_P^2-2V_S^2) \\
%       \mu    = \rho V_S^2 \\
%       \rho_0 = \rho 
%     \end{array} \right.\]
%     
%     \begin{eqnarray}
%       \nabla_{V_P} &=& \nabla_{\kappa} \partderi{\kappa}{V_P} + \nabla_{\rho_0} \partderi{\rho_0}{V_P} \nonumber\\
%                    &=& \nabla_{\kappa} 2\rho V_P \nonumber\\
%       \nabla_{V_S} &=& \nabla_{\kappa} \partderi{\kappa}{V_P} + \nabla_{\rho_0} \partderi{\rho_0}{V_P} \nonumber\\
%                    &=& \nabla_{\kappa} 2\rho V_P \nonumber\\
%       \nabla_{\rho} &=& \nabla_{\kappa} \partderi{\kappa}{\rho} + \nabla_{\rho_0} \partderi{\rho_0}{\rho} \nonumber\\
% 		    &=& \nabla_{\kappa} V_P^2 + \nabla_{\rho_0} \nonumber
%     \end{eqnarray}
% 
%   \end{minipage} \vline
%   \begin{minipage}{0.4\linewidth}
%     \[\left\{ \begin{array}{l}
%       V_P  = \sqrt{\frac{\lambda+2\mu}{\rho_0}} \\
%       V_S  = \sqrt{\mu/\rho_0} \\
%       \rho = \rho_0
%     \end{array} \right.\]
%     
%     \begin{eqnarray}
%       \nabla_{\kappa} &=& \nabla_{V_P} \partderi{V_P}{\kappa} + \nabla_{\rho} \partderi{\rho}{\kappa} \nonumber\\
%                       &=& \nabla_{V_P} 0.5(\kappa/\rho)^{0.5}/\rho \nonumber\\
%       \nabla_{\rho_0} &=& \nabla_{V_P} \partderi{V_P}{\rho_0} + \nabla_{\rho} \partderi{\rho}{\rho_0} \nonumber\\
%                       &=& \nabla_{V_P} (-0.5)(\kappa/\rho)^{0.5} \rho^{-2} + \nabla_{\rho} \nonumber
%     \end{eqnarray}
%   \end{minipage}
% 
% \end{frame}

\end{document}



%%%%%%%%%%%%%%%%%%%%%%%%%%%%
%%%%%% GOOD THINGS %%%%%%%%%
% %-------------------------------------------------------------------------------------
% \frame{
%  %\MyLogo
% \frametitle{}
% 
% \vspace*{\fill} 
% \begin{minipage}[c][8cm]{5cm}
% \begin{itemize}
% \item Source in red
% \item Receiver in blue
% \end{itemize}
% \end{minipage}
% 
% \begin{block}{Algorithm}
% %\footnotesize{
% \begin{algorithmic}[1]
% %%-----------------------------------------------
% \FOR {$frequency = frequency\_1$ to $frequency\_n$}
% \WHILE {(NOT $convergence$ AND $iter<niter_{max}$)}
% \STATE Estimate source wavelet if required
% \STATE Build gradient vector ${\mathcal{G}}_{\bf{m}}^{(k)}$
% \STATE Build perturbation vector $\delta \bf{m}$
% \STATE Update model ${\bf{m}}^{(k+1)} = {\bf{m}}^{(k)} + \alpha  \delta {\bf{m}}$
% \ENDWHILE
% \ENDFOR
% %%-----------------------------------------------
% \label{algo1}
% \end{algorithmic}
% %\end{algorithm}
% %}
% \end{block}
% }
%
% \colorbox{yellow}{reflection data}
%
% % Fontsize
%     \tiny
%     \scriptsize
%     \footnotesize
%     \small
%     \normalsize
%     \large
%     \Large
%     \LARGE
%     \huge
%     \Huge 
%
% % Multiple equations 
% \begin{eqnarray}
% \delta C & = &  p \\
%          & = &  p \rangle
% \label{eq}
% \end{eqnarray}
%add multiple eq. indices
%
% \[ f(n) = \left\{
%   \begin{array}{l l}
%          n/2 & \quad \text{if $n$ is even}\\
%     -(n+1)/2 & \quad \text{if $n$ is odd}
%   \end{array} \right.\]
%can't use {equation} because it's an array (can't add eq. index)
%
%\begin{equation}
%  u(x) =
%   \begin{cases}
%    \exp{x} & \text{if } x \geq 0 \\
%    1       & \text{if } x < 0
%   \end{cases}
%\end{equation}
%can use {equation} because it's an equation (can add eq. index)
%
% % Split an equation into multiple lines
%   \begin{equation}
%   \begin{split}
%     f(x) & =  1 \\
% 	  & -  g(x)
%   \end{split}
%   \end{equation}
% %-------------------------------------------------------------------------------------
%%%%%%%%%%%%%%%%%%%%%%%%%%%%


